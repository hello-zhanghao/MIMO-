\documentclass[12pt,a4paper]{ctexart}  % 选择文档类型
% \usepackage{latexsym}
\usepackage{amsmath}   % aligned必须需要该宏宝
\usepackage[colorlinks,linkcolor=red]{hyperref}  % 超链接宏包
\usepackage[ruled]{algorithm2e}   % 伪代码专用宏包
\usepackage{cite}  % 文献引用宏包
\usepackage{tikz}  % Tikz绘图包,基于底层的pgf
\usepackage{graphicx}      % 插入图片引用宏包
\usepackage{listingsutf8}  % 插入代码使用的宏包
\usepackage{color}         % 颜色宏包  
\usepackage{subfigure}     % 图片并排排列
\usepackage{bm}%专门处理数学粗体的bm宏包  
\usepackage{amsmath, amssymb}  % 
\usepackage{mathrsfs}
% \usetikzlibrary{fadings}

% \tikzfading[name=fade out, inner color=transparent!0, outer color=transparent!100]


\graphicspath{{figures/}}  % 指定图片目录, 与graphicx宏包一起使用
\setmainfont{Times New Roman}  % 设置英文字体 
\lstset{
    % backgroundcolor=\color{red!50!green!50!blue!50},%代码块背景色为浅灰色
    rulesepcolor= \color{gray}, %代码块边框颜色
    breaklines=true,  %代码过长则换行
    numbers=left, %行号在左侧显示
    numberstyle= \small,%行号字体
    keywordstyle= \color{blue},%关键字颜色
    commentstyle=\color{gray}, %注释颜色
    frame=shadowbox%用方框框住代码块
}
% \usepackage{algorithm}
% \usepackage{algpseudocode}
% \usepackage{amsmath}
% \renewcommand{\algorithmicrequire}{\textbf{Input:}}  % Use Input in the format of Algorithm
% \renewcommand{\algorithmicensure}{\textbf{Output:}} % Use Output in the format of Algorithm

\author{zfzdr}      % 作者
\title{MIMO技术}    % 题目
\date{2020年9月15日起}


% \frenchspacing
\begin{document}
\pagestyle{plain}    % 文档风格
\maketitle           % 显示标题
\tableofcontents     % 显示目录
% \input{contents/MIMO技术背景.tex}
% \input{contents/MIMO常见系统框架.tex}
% \section{MIMO信道}
\subsection{CDL信道}
\begin{lstlisting}[language={matlab}]
    % DelayProfile: 时延谱,可选'CDL-A' (default) | 'CDL-B' | 'CDL-C' | 'CDL-D' | 'CDL-E' | 'Custom'
    %----------------------------------------------------------------
    % 当DelayProfile为Custom时,下述参数有效
    % AveragePathGains — Average path gains in dB
    % AnglesAoA — Azimuth of arrival angle in degrees
    % AnglesAoD — Azimuth of departure angle in degrees
    % AnglesZoA — Zenith of arrival angle in degrees
    % AnglesZoD — Zenith of departure angle in degrees
    % HasLOSCluster — Line of sight cluster of the delay profile
    % KFactorFirstCluster — K-factor in first cluster of delay profile in dB
    % AngleSpreads — Scaled or cluster-wise RMS angle spreads in degrees(to enable this property, set DelayProfile to 'Custom' or AngleScaling to true.)
    % XPR — Cross-polarization power ratio in dB
    % NumStrongestClusters — Number of strongest clusters to split into subclusters 0 (default) | numeric scalar
    % ClusterDelaySpread — Cluster delay spread in seconds
    %-------------------------------------------------------------------
    % AngleScaling — Apply scaling of angles
    % false (default) | true
    % 当DelayProfile为true时,下述参数有效
    % MeanAngles — Scaled mean angles in degrees [0.0 0.0 0.0 0.0] (default) | four-element row vector
    %------------------------------------------------------------------
    % DelaySpread — Desired RMS delay spread in seconds (Use this property to
    % specify the delay offset between subclusters for clusters split into subclusters. See TR 38.901 Section 7.5, Step 11.)
    % CarrierFrequency — Carrier frequency in Hz 4e9 (default) | numeric scalar
    % MaximumDopplerShift — Maximum Doppler shift in Hz 5 (default) | nonnegative numeric scalar
    % UTDirectionOfTravel — User terminal direction of travel in degrees
    % KFactorScaling — K-factor scaling false (default) | true
    % KFactor — Desired K-factor for scaling in dB 9.0 (default) | numeric scalar
    % SampleRate — Sample rate of input signal in Hz 30.72e6 (default) | positive numeric scalar
    % TransmitAntennaArray — Transmit antenna array characteristics structure
    % ReceiveAntennaArray — Receive antenna array characteristics structure
    % SampleDensity — Number of time samples per half wavelength 64 (default) | Inf | numeric scalar
    % NormalizePathGains — Normalize path gains true (default) | false
    % InitialTime — Time offset of fading process in seconds 0.0 (default) | numeric scalar
    % RandomStream — Source of random number stream
    % Seed — Initial seed of mt19937ar random number stream
    % ChannelFiltering — Fading channel filtering
    % NumTimeSamples — Number of time samples
    % NormalizeChannelOutputs — Normalize channel outputs by the number of receive antennas
    % 信道估计返回值
    % Perfect channel estimate, returned as an NSC-by-NSYM-by-NR-by-NT complex array, where:
    % 
    % NSC is the number of subcarriers.
    % 
    % NSYM is the number of OFDM symbols.
    % 
    % NR is the number of receive antennas.
    % 
    % NT is the number of transmit antennas.
    
    
    clear;
    v = 15.0;                    % UT velocity in km/h
    fc = 4e9;                    % carrier frequency in Hz
    c = physconst('lightspeed'); % speed of light in m/s
    fd = (v*1000/3600)/c*fc;     % UT max Doppler frequency in Hz
    
    % CDL配置
    cdl = nrCDLChannel;  % creat CDL对象
    cdl.DelayProfile = 'CDL-D';    % 选择典型信道场景
    cdl.DelaySpread = 10e-9;       % 时延扩展因子
    cdl.CarrierFrequency = fc;     % 载波频率,
    cdl.MaximumDopplerShift = fd;  % 最大多普勒频偏
    cdl.TransmitAntennaArray.Size = [2 2 1 1 1];  % 发送天线配置
    cdl.ReceiveAntennaArray.Size = [2 2 2 1 1];   % 接收天线配置
    
    % 发送信号,Create a random waveform of 1 subframe duration with 8 antennas.
    SR = 15.36e6;  
    T = SR * 1e-3;  % 一个子帧(1ms)信号采样数
    cdl.SampleRate = SR;  % 输入信号采样率
    cdlinfo = info(cdl);
    Nt = cdlinfo.NumTransmitAntennas;
    txWaveform = complex(randn(T,Nt),randn(T,Nt)); % 发送信号
    
    % 过信道
    [~,pathGains,sampleTimes] = cdl(txWaveform);
    pathFilters = getPathFilters(cdl);
    
    % 信道估计及可视化
    NRB = 100;  % 资源块
    SCS = 120;  % 子载波间隔,15 | 30 | 60 | 120 | 240 可选
    nSlot = 0;
    offset = nrPerfectTimingEstimate(pathGains,pathFilters);
    hest = nrPerfectChannelEstimate(pathGains,pathFilters, NRB,SCS,nSlot,offset,sampleTimes);
    size(hest)
    hest(1, 1, 1, 1)
    figure;
    surf(abs(hest(:,:,2, 1)).^2);
    shading('flat');
    xlabel('OFDM Symbols');
    ylabel('Subcarriers');
    zlabel('|H|');
    title('Channel Magnitude Response');
    
    x = size(hest);
    tj = zeros(x(1)*x(2), 1);
    for i = 1:x(1)
        for j = 1:x(2)
            tj((i-1)*x(2)+j) = cond(squeeze(hest(i,j, :, :)));
        end
    end
    histogram(tj, 100)    
\end{lstlisting}
% \section{MIMO信道估计}

% \input{contents/MIMO信道反馈.tex}
\part{MIMO检测}
\section{点对点MIMO检测}
\subsection{系统模型}
$$
y=Hx+z
$$

% \begin{enumerate}
%     \item  参考论文\cite{2019Massive} \par
%         MIMO综述,主要综述了关于MIMO检测的问题
%     \item 参考论文2
% \end{enumerate}
\subsection{经典检测方法}
% 假设接收端已知CSI $H^{UL}$,系统模型表示为:
% \begin{equation}
% y_{MAC}=H^{UL}x+z 简化为 y=Hx+z=H_1x_1+H_2x_2+\cdots+H_Kx_K+z
% \end{equation}
\subsubsection{MF检测}
\begin{equation}
\hat{x}_{MF}=H^Hy
\end{equation}
匹配滤波器,又称最大比合并,目标是最大化接收信噪比。
\begin{itemize}
    \item 优点:复杂度低,不需要矩阵求逆运算
    \item 缺点:对于病态矩阵(ill-conditioned),检测性能严重劣化
\end{itemize}
% \end{enum}
    
\subsubsection{ZF检测}
\begin{equation}
    \hat{x}_{ZF}=G_{ZF}y
\end{equation}
其中$G_{ZF}=(H^HH)^{-1}H^H$,目标是最大化接收信干噪比(received signal-to-interference ration)
\begin{itemize}
    \item 优点:性能优于MF(复杂度较MF高)
    \item 缺点:对忽略噪声的影响,与MF类似,会有噪声加强的副作用
\end{itemize}
\subsubsection{MMSE检测}
MMSE检测是优化问题的解:
\begin{equation}
\begin{aligned}
    G_{MMSE}&=\underset{G\in C^{N_t \times N_r}}{\arg min}\|x-Gy\|^2 \\
    &= \underset{G}{\arg min}E\{\|Gy-x\|^2\}=(H^HH+\sigma^2I_{Nt})^{-1}H^H
\end{aligned}
\end{equation}
于是,MMSE检测表达式为:
\begin{equation}
    \hat{x}_{MMSE}=G_{MMSE}y
\end{equation} \\

\emph{regularized channel inversion}
\begin{equation}
    \underline{H}=\begin{bmatrix}
        H \\
        \sigma^2I_{N_t}
        \end{bmatrix} and  \ \ 
        \underline{y}=
        \begin{bmatrix}
        y \\
        0_{N_t\times1}
        \end{bmatrix} 
\end{equation}
Then
\begin{equation}
    G_{MMSE}=(\underline{H}^H\underline{H})^{-1}\underline{H}^H
\end{equation}
\begin{itemize}
    \item 优点:有效解决了ZF和MF导致的噪声加强问题,能够在噪声功率较大(信噪比较低时)时获得较大的性能增益
    \item 缺点:需要对噪声功率谱密度的先验信息,且需要矩阵求逆
\end{itemize}

\subsubsection{ZF-SIC检测}


\subsubsection{MMSE-SIC检测}
\subsubsection{总结}
最早的大规模MIMO检测器是在2008年由Vardhan et al. \cite{Vishnu2008A}提出的,是基于最大似然搜索的。紧接着,研究者提出了使用局部搜索和置信传播提出了近似ML的检测器。但是随着天线数目的增加,矩阵求逆操作的复杂度呈指数增长,计算复杂度急剧上升,为了解决这个问题,2013年,
Wu et al.\cite{2013Approximate} 提出了基于近似求逆方法的上行检测器。

\subsection{Massive MIMO 检测器}
\subsubsection{基于近似求逆的线性检测器}
\subsubsection{Newton Iteration Method}
\subsubsection{Gauss-Seidel Method}
\subsubsection{Successive Over-Relaxation}
\subsubsection{Jacobi Method}
\subsubsection{Richardson Method}
\subsubsection{Conjugate Gradients Method}
\subsubsection{Lanczos Method}
\subsubsection{Residual Method}
\subsubsection{Coordinate Descent Method}
\subsection{基于局部搜索的检测器}
\subsubsection{Likelihood Ascent Search}
\subsubsection{Reactive Tabu Search}
\subsection{基于置信传播的检测器}
\subsection{BOX Detection}
\subsection{Sparsity Based Algorithms}
\subsection{小尺度MIMO检测器在大尺度MIMO检测中的应用}
\subsubsection{Successive Interference Cancellation}
\subsubsection{Lattice Reduction-Aided Algorithms}
\subsubsection{Sphere Decoder}
\subsection{基于AI的MIMO检测方法}

\section{MIMO预编码}
\subsection{点对点MIMO系统模型}
\begin{equation}
    y=Hx+z
\end{equation}
其中,$y\in C^{N_r\times1}$,$x\in C^{N_t\times 1}$,$z\in C^{N_r\times1}$,发送端总能量为$E\{x^Hx\}=P$,噪声功率谱密度为$N_0$,即$E\{zz^H\}=N_0I_{N_r}$,且
\begin{equation}
    \begin{aligned}
        R_{yy}&=E\{yy^H\} \\
        &=HR_{xx}H^H+N_0I_{N_r}
    \end{aligned}
\end{equation}

\subsection{系统容量}
\begin{equation}
    \begin{aligned}
        I(x;y)&=H(x)-H(x|y)\\
        &=H(y)-H(y|x) \\
        &=H(y)-H(Hx+z|x) \\
        &=H(y)-H(z|x) \\
        &=H(y)-H(z)
    \end{aligned}
\end{equation}
其中,$z$是满足复高斯随机分布的多维向量,因此当且仅当$y$也满足复高斯随机分布时,上式取得最大值,且
\begin{equation}
\begin{aligned}
    H(y)&=log_2|\pi eR_{yy}| =log_2|\pi eHR_{xx}H^H+\pi eN_0I_{N_r}| \\
    H(z)&=log_2|\pi e N_0I_{N_r}|
\end{aligned}
\end{equation}
于是,
\begin{equation}
    I(x;y)=log_2\left|I_{N_r} + \frac{HR{xx}H^H}{N_0}\right|
\end{equation}

\subsection{传输侧无CSI}
假设每根天线上的发送信号能量相等且相互独立,即$R_{xx}=\frac{P}{N_t}I_{N_t}$,则
\begin{equation}
    \begin{aligned}
    C&=log_2\left|I_{N_r} + \frac{P}{N_tN_0}HH^H\right| \\
    &=\sum_{i=1}^{N_t}log_2(1+\frac{P}{N_tN_0}\lambda_i)
    \end{aligned}
\end{equation}

\subsection{传输侧有CSI}
预编码提高信道容量 \\
对信道矩阵$H$使用SVD分解,即$H=U\Sigma V^H$,一般假设$Nr>Nt$,则
\begin{equation}
\Sigma=\left[ 
    \begin{matrix}
        \sqrt{\lambda_1} & 0 & \cdots & 0 \\
        0 & \sqrt{\lambda_2} & \cdots & 0 \\
        \vdots & \vdots & \ddots & \vdots \\
        0 & 0 & \cdots & \sqrt{\lambda_{N_t}} \\
        \vdots & \vdots & \vdots & \vdots \\
        0 & 0 & 0 & 0
    \end{matrix}
\right] 
\end{equation}
令调制后信号能量表示为$\tilde{x}$,预编码后的发送信号能量为$x=V^H\tilde{x}$,则

\begin{equation}
    \begin{aligned}
    y &= Hx+z \\
    &=U\Sigma V^HV\tilde{x}+z \\ 
    &=U\Sigma\tilde{x}+z
    \end{aligned}
\end{equation}  
\begin{equation}
    U^Hy = U^HU\Sigma\tilde{x}+U^Hz =>
    \tilde{y}=\Sigma\tilde{x}+\tilde{z}
\end{equation} 


上式展开为
\begin{equation}
\left[
    \begin{matrix}
        \tilde{y}_1 \\
        \tilde{y}_2 \\
        \vdots \\
        \tilde{y}_{N_t} \\
        \vdots \\
        \tilde{y}_{N_r}
    \end{matrix}
\right]
=\left[ 
\begin{matrix}
    \sqrt{\lambda_1} & 0 & \cdots & 0 \\
    0 & \sqrt{\lambda_2} & \cdots & 0 \\
    \vdots & \vdots & \ddots & \vdots \\
    0 & 0 & \cdots & \sqrt{\lambda_{N_t}} \\
    \vdots & \vdots & \vdots & \vdots \\
    0 & 0 & 0 & 0
\end{matrix}
\right]
\left[
\begin{matrix}
    \tilde{x}_1 \\
    \tilde{x}_2 \\
    \vdots \\
    \tilde{x}_{N_t}
    \end{matrix}
    \right] + 
    \left[
\begin{matrix}
    \tilde{z}_1 \\
    \tilde{z}_2 \\
    \vdots \\
    \tilde{z}_{N_t} \\
    \vdots \\
    \tilde{z}_{N_r}
\end{matrix}
\right]
\end{equation}

即
\begin{equation}
    \tilde{y}_i=\sqrt{\lambda_i}\tilde{x}_i+\tilde{z}_i, i=1,\cdots,r.一般\ r=N_t
\end{equation}
原始的MIMO信道等效为$r$个SISO信道,每个SISO信道的信道容量可以表示为:
\begin{equation}
    C_i(P_i)=log_2(1+\frac{\lambda_iP_i}{N_0})
\end{equation}
其中,$P_i$表示第$i$根天线上的信号能量,且
\begin{equation}
    E\{x^Hx\}=\sum_{i=1}^{N_t}E\{|x_i|^2\}=\sum_{i=1}^{N_t}P_i=P
\end{equation}
于是,信道总容量为:
\begin{equation}
    C=\sum_{i=1}^{N_t}C_i(P_i)=\sum_{i=1}^{N_t}log_2(1+\frac{\lambda_iP_i}{N_0})
\end{equation}
可以通过注水算法优化功率分配,达到更大的信道容量,即
\begin{equation}
    \begin{aligned}
        & C={\underset{\{P_i\}} {\operatorname {arg\,max}}}\ \sum_{i=1}^{N_t}C_i(P_i)=\sum_{i=1}^{N_t}log_2(1+\frac{\lambda_iP_i}{N_0}) \\
        & s.t\ \  \sum_{i=1}^{N_t}P_i=P 
    \end{aligned}
\end{equation}
最优解为   
\begin{equation}
    \begin{aligned}
        P_i^{opt}&=(\mu -\frac{N_0}{\lambda_i})^+, i=1,\cdots,r \\
        \sum_{i=1}^{N_t}P_i&=P
    \end{aligned}
\end{equation}

\subsection{MIMO注水算法}
\begin{algorithm}
    \caption{注水算法}
    Step1:迭代计算p=1,计算$\mu=\frac{N_t}{r-\rho+1}$ \\
    Step2:用$\mu$计算$\gamma_i=\mu-\frac{N_tN_0}{E_x\lambda_i},i=1,2,\cdots,r-p+1$ \\
    Step3:若分配到最小增益的信道能量为负值,即设$\gamma_{r-p+1}=0,p=p+1$,转至Step1 \\
    若任意$\gamma_i$非负,即得到最佳注水功率分配策略
\end{algorithm}

% \section{AMP算法}
\href{https://krzakala.github.io/cargese.io/AMP_Tutorial_18.pdf}{参考资料}
\subsection{Model}
\begin{eqnarray}
    y=Ax+w
\end{eqnarray}
问题{\color{red}{怎么从y和A估计x,使得x的估计和x间在某一准则下最小}}
\subsection{Unconstrained Least Squares Estimation}
\begin{equation}
    \hat{x}=\underset{x}{\arg min}\frac{1}{2}\|y-Ax\|^2=(A^TA)^{-1}A^Ty
\end{equation}
\subsection{Regularized Least Squares Estimation}
\begin{equation}
    \hat{x}=\underset{x}{\arg min}\frac{1}{2}\|y-Ax\|^2+{\color{purple}{\phi(x)}}
\end{equation}
\begin{itemize}
    \item L2 $\phi(x)=\lambda\|x\|^2_2$
    \item L1 $\phi(x)=\lambda\|x\|_1$
\end{itemize}
\subsection{Proximal Operators}
\subsection{Majorization-Minimization}
\begin{itemize}
    \item 假设最小化$f(x)$很难直接求解
    \item 在每个点$x_k$,可以找到一个函数majorizing function$Q(x,x_k)$:
    \begin{itemize}
        \item $Q(x,x_k)\geq for\ all\ x$
        \item $Q(x_k,x_k)=f(x_k)$
    \end{itemize}
    \item Majorization-Minimization algorithm: \par 
    Iteratively minimize the majorizing function:$x_{k+1}=\underset{x}{\arg min}Q(x,x_k)$
    \item {\color{purple}{Theorem}}: MM monotonically decreases the true objective: 
    $$f(x_k)=Q(x_k,x_k)\geq Q(x_{k+1},x_k)\geq f(x_{k+1})$$
\end{itemize}
\subsection{MM for Regularized LS}
RLS可以重写为:
\begin{equation}
    \hat{x}=\underset{x}{\arg min}[g(x)+\phi(x)]
\end{equation}
其中$g(x)=\frac{1}{2}\|y-Ax\|^2$
定义majorizing 函数为:
\begin{equation}
    Q(x,x_k)=g(x_k)+\nabla g(x_k)\cdot (x-x_k)+\frac{1}{2\alpha}\|x-x_k\|^2+\phi(x)
\end{equation}
很容易证明
\begin{itemize}
    \item $Q(x_k,x_k)=g(x_k)+h(x_k)=f(x_k)$
    \item $Q(x,x_k)\geq f(x_k) for all x()$当$\alpha$足够小
\end{itemize}
因此,\begin{equation}
    x_{k+1}=\underset{x}{\arg min}Q(x,x_k)
\end{equation}
\subsection{ISTA Algorithm}
\begin{equation}
    \begin{aligned}
    x_{k+1}&=\underset{x}{\arg min}Q(x,x_k) \\
    &=\underset{x}{\arg min}\ g(x_k)+\nabla g(x_k)  \cdot  (x-x_k)+\frac{1}{2\alpha}\|x-x_k\|^2+\phi(x)
    \end{aligned}
\end{equation}
{\color{purple}{Iterative Soft Threshold Algorithm:}}
\begin{itemize}
    \item Gradient : $r_k=x_k-\alpha \nabla g(x_k)=x_k-\alpha A^T(Ax_k-y)$
    \item Proximal step: $x_{k+1}=\text{Prox}_{\phi}(r_k,\alpha)=\underset{x}{\arg min}\ [\phi(x)+\frac{1}{2\alpha}\|r_k-x\|^2] $
\end{itemize}
理解Prox操作的意义
\subsection{AMP}
\begin{figure}
    \includegraphics[width=0.99\textwidth]{AMP和ISTA.PNG}
\end{figure}
% \section{多用户MIMO上行系统}
\subsection{基于AI的多用户MIMO检测}
\subsubsection{DeepSIC\cite{2020DeepSIC}}
\href{https://github.com/nirshlezinger1/DeepSIC}{源码} \par 
\paragraph{缺点}
文章只探究了固定信道的性能,且网络中没有利用信道矩阵H的信息,很难在时变信道情况下应用该网络,文章提到了在线训练的方法,但是该方法和我们研究的直接使用信道矩阵和接收信号进行检测有一定的差异性。
\paragraph{系统模型}
\begin{equation}
    y=Hx+n
\end{equation}
其中,$H_{i,j}=e^{-|i-j|},i\in \{1,\cdots,N_r\}, j\in \{1,\cdots, K\}$
\paragraph{核心思想}\par 
利用传统的迭代干扰消除的思想,每一次迭代将已检测出的信号当作干扰从接收信号中消除,然后检测剩下的符号。传统的干扰消除的消除方法是直接将其余估计信号从接收信号中减去,然后估计待估计信号。其原理如图所示: \par 
\includegraphics[width=0.99\textwidth]{SIIC.PNG}
% \begin{figure}[ht]
%     \centering
%     \includegraphics[width=0.99\textwidth]{SIIC.PNG}
% \end{figure}
因为传统的干扰消除模块有一定的局限性,一是将信道视为线性信道模型,不适用于非线性信道,二是需要获取信道矩阵以及信道噪声的准确信息。为了解决这个问题,文章提出将传统的干扰消除模块用一个DNN网络来代替,通过数据训练的方法获取最优的基于网络的干扰消除模块,将该网络称为DeepSIC。
\includegraphics[width=0.9\textwidth]{分类DNN.PNG}
\includegraphics[width=0.9\textwidth]{DeepSIC框图.PNG}
\paragraph{损失函数}
\begin{equation}
    L_{SumCE}(\bm{\theta})=\frac{1}{n_t}\sum_{j=1}^{n_t}\sum_{k=1}^{K}-\log \hat{\bm{p}}_k^{(Q)}(\tilde{\bm{y}}_j,(\tilde{\bm{s}}_j)_k;\bm{\theta})
\end{equation}
\subsubsection{MLD+DCNN\cite{2020A}}
\paragraph{系统模型}
\begin{equation}
    \begin{aligned}
    \bm{y}(n)&=\bm{H}(n)\bm{s}(n)+\bm{w}(n) \\ 
    \bm{w}(n+1)&=\sqrt{\rho}\bm{w}(n)+\sqrt{1-\rho}\bm{u}(n+1)
    \end{aligned}
\end{equation}
\paragraph{核心思想} 
文章针对时域相关的噪声环境提出了MLD+DCNN的网络结构,其中MLD用于符号检测,DCNN用于噪声估计,通过MLD和DCNN结构将时域相关噪声解相关为时域不相关的噪声,从而使得MLD检测获得更优的性能。
\includegraphics[width=0.9\textwidth]{MLD_DCNN.PNG}

\subsubsection{MMNet}
\subsubsection{OAMPNet}
\subsubsection{DetNet}

% \begin{figure}
%     \subfigure{
%         \centering
%         \includegraphics[width=0.9\textwidth]{分类DNN.PNG}
%         }
%     \subfigure{
%         \centering
%         \includegraphics[width=0.9\textwidth]{DeepSIC框图.PNG}
%     }
% \end{figure}
% \begin{enumerate}
%     \item DeepSIC\cite{2020DeepSIC} \par 
%         系统模型$y=Hx+n$,$H_{i,j}=e^{-|i-j|},i\in \{1,\cdots,N_r\}, j\in \{1,\cdots, K\}$ \par 
%     \item 
% \end{enumerate}
\section{Latex语法学习}
\subsection{伪代码}
\href{伪代码}{http://hustsxh.is-programmer.com/posts/38801.html}
\paragraph{algorithmic和algorithmics}
algorithmic和algorithmicx,这两个包很像,很多命令都是一样的,只是algorithmic的命令都是大写,algorithmicx的命令都是首字母大写,其他小写(EndFor两个大写)。下面是algorithmic的基本命令
还有algorithm2e,latex的与algorithm相关的包常用的有几个,algorithm、algorithmic、algorithmicx、algorithm2e,可以大致分成三类,或者说三个排版环境。最原始的是使用algorithm+algorithmic,这个最早出现,也是最难用的,需要自己定义一些指令。第二个排版环境是algorithm+algorithmicx,algorithmicx提供了一些宏定义和一些预定义好了的环境(layout),指令类似algorithmic。第三个是algorithm2e,只需要一个包,使用起来和编程的感觉很像,也是我更倾向使用的包。下面是使用algorithm2e的例子。[1]
\subsection{超链接}
需要的宏包 hyperref, 


% 参考文献
\bibliographystyle{plain}      % 设置引用格式的风格
\bibliography{reference.bib}   % 引用自己的bib文件,保证在当前文件的目录下
\end{document}