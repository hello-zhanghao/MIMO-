\section{MIMO预编码}
\subsection{点对点MIMO系统模型}
\begin{equation}
    y=Hx+z
\end{equation}
其中,$y\in C^{N_r\times1}$,$x\in C^{N_t\times 1}$,$z\in C^{N_r\times1}$,发送端总能量为$E\{x^Hx\}=P$,噪声功率谱密度为$N_0$,即$E\{zz^H\}=N_0I_{N_r}$,且
\begin{equation}
    \begin{aligned}
        R_{yy}&=E\{yy^H\} \\
        &=HR_{xx}H^H+N_0I_{N_r}
    \end{aligned}
\end{equation}

\subsection{系统容量}
\begin{equation}
    \begin{aligned}
        I(x;y)&=H(x)-H(x|y)\\
        &=H(y)-H(y|x) \\
        &=H(y)-H(Hx+z|x) \\
        &=H(y)-H(z|x) \\
        &=H(y)-H(z)
    \end{aligned}
\end{equation}
其中,$z$是满足复高斯随机分布的多维向量,因此当且仅当$y$也满足复高斯随机分布时,上式取得最大值,且
\begin{equation}
\begin{aligned}
    H(y)&=log_2|\pi eR_{yy}| =log_2|\pi eHR_{xx}H^H+\pi eN_0I_{N_r}| \\
    H(z)&=log_2|\pi e N_0I_{N_r}|
\end{aligned}
\end{equation}
于是,
\begin{equation}
    I(x;y)=log_2\left|I_{N_r} + \frac{HR{xx}H^H}{N_0}\right|
\end{equation}

\subsection{传输侧无CSI}
假设每根天线上的发送信号能量相等且相互独立,即$R_{xx}=\frac{P}{N_t}I_{N_t}$,则
\begin{equation}
    \begin{aligned}
    C&=log_2\left|I_{N_r} + \frac{P}{N_tN_0}HH^H\right| \\
    &=\sum_{i=1}^{N_t}log_2(1+\frac{P}{N_tN_0}\lambda_i)
    \end{aligned}
\end{equation}

\subsection{传输侧有CSI}
预编码提高信道容量 \\
对信道矩阵$H$使用SVD分解,即$H=U\Sigma V^H$,一般假设$Nr>Nt$,则
\begin{equation}
\Sigma=\left[ 
    \begin{matrix}
        \sqrt{\lambda_1} & 0 & \cdots & 0 \\
        0 & \sqrt{\lambda_2} & \cdots & 0 \\
        \vdots & \vdots & \ddots & \vdots \\
        0 & 0 & \cdots & \sqrt{\lambda_{N_t}} \\
        \vdots & \vdots & \vdots & \vdots \\
        0 & 0 & 0 & 0
    \end{matrix}
\right] 
\end{equation}
令调制后信号能量表示为$\tilde{x}$,预编码后的发送信号能量为$x=V^H\tilde{x}$,则

\begin{equation}
    \begin{aligned}
    y &= Hx+z \\
    &=U\Sigma V^HV\tilde{x}+z \\ 
    &=U\Sigma\tilde{x}+z
    \end{aligned}
\end{equation}  
\begin{equation}
    U^Hy = U^HU\Sigma\tilde{x}+U^Hz =>
    \tilde{y}=\Sigma\tilde{x}+\tilde{z}
\end{equation} 


上式展开为
\begin{equation}
\left[
    \begin{matrix}
        \tilde{y}_1 \\
        \tilde{y}_2 \\
        \vdots \\
        \tilde{y}_{N_t} \\
        \vdots \\
        \tilde{y}_{N_r}
    \end{matrix}
\right]
=\left[ 
\begin{matrix}
    \sqrt{\lambda_1} & 0 & \cdots & 0 \\
    0 & \sqrt{\lambda_2} & \cdots & 0 \\
    \vdots & \vdots & \ddots & \vdots \\
    0 & 0 & \cdots & \sqrt{\lambda_{N_t}} \\
    \vdots & \vdots & \vdots & \vdots \\
    0 & 0 & 0 & 0
\end{matrix}
\right]
\left[
\begin{matrix}
    \tilde{x}_1 \\
    \tilde{x}_2 \\
    \vdots \\
    \tilde{x}_{N_t}
    \end{matrix}
    \right] + 
    \left[
\begin{matrix}
    \tilde{z}_1 \\
    \tilde{z}_2 \\
    \vdots \\
    \tilde{z}_{N_t} \\
    \vdots \\
    \tilde{z}_{N_r}
\end{matrix}
\right]
\end{equation}

即
\begin{equation}
    \tilde{y}_i=\sqrt{\lambda_i}\tilde{x}_i+\tilde{z}_i, i=1,\cdots,r.一般\ r=N_t
\end{equation}
原始的MIMO信道等效为$r$个SISO信道,每个SISO信道的信道容量可以表示为:
\begin{equation}
    C_i(P_i)=log_2(1+\frac{\lambda_iP_i}{N_0})
\end{equation}
其中,$P_i$表示第$i$根天线上的信号能量,且
\begin{equation}
    E\{x^Hx\}=\sum_{i=1}^{N_t}E\{|x_i|^2\}=\sum_{i=1}^{N_t}P_i=P
\end{equation}
于是,信道总容量为:
\begin{equation}
    C=\sum_{i=1}^{N_t}C_i(P_i)=\sum_{i=1}^{N_t}log_2(1+\frac{\lambda_iP_i}{N_0})
\end{equation}
可以通过注水算法优化功率分配,达到更大的信道容量,即
\begin{equation}
    \begin{aligned}
        & C={\underset{\{P_i\}} {\operatorname {arg\,max}}}\ \sum_{i=1}^{N_t}C_i(P_i)=\sum_{i=1}^{N_t}log_2(1+\frac{\lambda_iP_i}{N_0}) \\
        & s.t\ \  \sum_{i=1}^{N_t}P_i=P 
    \end{aligned}
\end{equation}
最优解为   
\begin{equation}
    \begin{aligned}
        P_i^{opt}&=(\mu -\frac{N_0}{\lambda_i})^+, i=1,\cdots,r \\
        \sum_{i=1}^{N_t}P_i&=P
    \end{aligned}
\end{equation}

\subsection{MIMO注水算法}
\begin{algorithm}
    \caption{注水算法}
    Step1:迭代计算p=1,计算$\mu=\frac{N_t}{r-\rho+1}$ \\
    Step2:用$\mu$计算$\gamma_i=\mu-\frac{N_tN_0}{E_x\lambda_i},i=1,2,\cdots,r-p+1$ \\
    Step3:若分配到最小增益的信道能量为负值,即设$\gamma_{r-p+1}=0,p=p+1$,转至Step1 \\
    若任意$\gamma_i$非负,即得到最佳注水功率分配策略
\end{algorithm}
