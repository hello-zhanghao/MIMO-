\section{MIMO信道}
\subsection{CDL信道}
\begin{lstlisting}[language={matlab}]
    % DelayProfile: 时延谱,可选'CDL-A' (default) | 'CDL-B' | 'CDL-C' | 'CDL-D' | 'CDL-E' | 'Custom'
    %----------------------------------------------------------------
    % 当DelayProfile为Custom时,下述参数有效
    % AveragePathGains — Average path gains in dB
    % AnglesAoA — Azimuth of arrival angle in degrees
    % AnglesAoD — Azimuth of departure angle in degrees
    % AnglesZoA — Zenith of arrival angle in degrees
    % AnglesZoD — Zenith of departure angle in degrees
    % HasLOSCluster — Line of sight cluster of the delay profile
    % KFactorFirstCluster — K-factor in first cluster of delay profile in dB
    % AngleSpreads — Scaled or cluster-wise RMS angle spreads in degrees(to enable this property, set DelayProfile to 'Custom' or AngleScaling to true.)
    % XPR — Cross-polarization power ratio in dB
    % NumStrongestClusters — Number of strongest clusters to split into subclusters 0 (default) | numeric scalar
    % ClusterDelaySpread — Cluster delay spread in seconds
    %-------------------------------------------------------------------
    % AngleScaling — Apply scaling of angles
    % false (default) | true
    % 当DelayProfile为true时,下述参数有效
    % MeanAngles — Scaled mean angles in degrees [0.0 0.0 0.0 0.0] (default) | four-element row vector
    %------------------------------------------------------------------
    % DelaySpread — Desired RMS delay spread in seconds (Use this property to
    % specify the delay offset between subclusters for clusters split into subclusters. See TR 38.901 Section 7.5, Step 11.)
    % CarrierFrequency — Carrier frequency in Hz 4e9 (default) | numeric scalar
    % MaximumDopplerShift — Maximum Doppler shift in Hz 5 (default) | nonnegative numeric scalar
    % UTDirectionOfTravel — User terminal direction of travel in degrees
    % KFactorScaling — K-factor scaling false (default) | true
    % KFactor — Desired K-factor for scaling in dB 9.0 (default) | numeric scalar
    % SampleRate — Sample rate of input signal in Hz 30.72e6 (default) | positive numeric scalar
    % TransmitAntennaArray — Transmit antenna array characteristics structure
    % ReceiveAntennaArray — Receive antenna array characteristics structure
    % SampleDensity — Number of time samples per half wavelength 64 (default) | Inf | numeric scalar
    % NormalizePathGains — Normalize path gains true (default) | false
    % InitialTime — Time offset of fading process in seconds 0.0 (default) | numeric scalar
    % RandomStream — Source of random number stream
    % Seed — Initial seed of mt19937ar random number stream
    % ChannelFiltering — Fading channel filtering
    % NumTimeSamples — Number of time samples
    % NormalizeChannelOutputs — Normalize channel outputs by the number of receive antennas
    % 信道估计返回值
    % Perfect channel estimate, returned as an NSC-by-NSYM-by-NR-by-NT complex array, where:
    % 
    % NSC is the number of subcarriers.
    % 
    % NSYM is the number of OFDM symbols.
    % 
    % NR is the number of receive antennas.
    % 
    % NT is the number of transmit antennas.
    
    
    clear;
    v = 15.0;                    % UT velocity in km/h
    fc = 4e9;                    % carrier frequency in Hz
    c = physconst('lightspeed'); % speed of light in m/s
    fd = (v*1000/3600)/c*fc;     % UT max Doppler frequency in Hz
    
    % CDL配置
    cdl = nrCDLChannel;  % creat CDL对象
    cdl.DelayProfile = 'CDL-D';    % 选择典型信道场景
    cdl.DelaySpread = 10e-9;       % 时延扩展因子
    cdl.CarrierFrequency = fc;     % 载波频率,
    cdl.MaximumDopplerShift = fd;  % 最大多普勒频偏
    cdl.TransmitAntennaArray.Size = [2 2 1 1 1];  % 发送天线配置
    cdl.ReceiveAntennaArray.Size = [2 2 2 1 1];   % 接收天线配置
    
    % 发送信号,Create a random waveform of 1 subframe duration with 8 antennas.
    SR = 15.36e6;  
    T = SR * 1e-3;  % 一个子帧(1ms)信号采样数
    cdl.SampleRate = SR;  % 输入信号采样率
    cdlinfo = info(cdl);
    Nt = cdlinfo.NumTransmitAntennas;
    txWaveform = complex(randn(T,Nt),randn(T,Nt)); % 发送信号
    
    % 过信道
    [~,pathGains,sampleTimes] = cdl(txWaveform);
    pathFilters = getPathFilters(cdl);
    
    % 信道估计及可视化
    NRB = 100;  % 资源块
    SCS = 120;  % 子载波间隔,15 | 30 | 60 | 120 | 240 可选
    nSlot = 0;
    offset = nrPerfectTimingEstimate(pathGains,pathFilters);
    hest = nrPerfectChannelEstimate(pathGains,pathFilters, NRB,SCS,nSlot,offset,sampleTimes);
    size(hest)
    hest(1, 1, 1, 1)
    figure;
    surf(abs(hest(:,:,2, 1)).^2);
    shading('flat');
    xlabel('OFDM Symbols');
    ylabel('Subcarriers');
    zlabel('|H|');
    title('Channel Magnitude Response');
    
    x = size(hest);
    tj = zeros(x(1)*x(2), 1);
    for i = 1:x(1)
        for j = 1:x(2)
            tj((i-1)*x(2)+j) = cond(squeeze(hest(i,j, :, :)));
        end
    end
    histogram(tj, 100)    
\end{lstlisting}