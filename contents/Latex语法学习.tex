\section{Latex语法学习}
\subsection{伪代码}
\href{伪代码}{http://hustsxh.is-programmer.com/posts/38801.html}
\paragraph{algorithmic和algorithmics}
algorithmic和algorithmicx,这两个包很像,很多命令都是一样的,只是algorithmic的命令都是大写,algorithmicx的命令都是首字母大写,其他小写(EndFor两个大写)。下面是algorithmic的基本命令
还有algorithm2e,latex的与algorithm相关的包常用的有几个,algorithm、algorithmic、algorithmicx、algorithm2e,可以大致分成三类,或者说三个排版环境。最原始的是使用algorithm+algorithmic,这个最早出现,也是最难用的,需要自己定义一些指令。第二个排版环境是algorithm+algorithmicx,algorithmicx提供了一些宏定义和一些预定义好了的环境(layout),指令类似algorithmic。第三个是algorithm2e,只需要一个包,使用起来和编程的感觉很像,也是我更倾向使用的包。下面是使用algorithm2e的例子。[1]
\subsection{超链接}
需要的宏包 hyperref, 

\subsection{插入图片}

\subsection{tikz绘图}
\begin{tikzpicture}
    \draw[red, -latex](0,0)--(5,5);
    \draw (5,5) rectangle (6,7);
    % \draw (5,5) circle (2, 3);
\end{tikzpicture}
% \mint{python}|test.py|		% 这里以Python语言为例。双竖线中是文件名
% \begin{minted}[mathescape,	% 中括号中的内容用于控制代码显示的格式,可以依照喜好修改
%                linenos,
%                numbersep=5pt,
%                gobble=2,
%                frame=lines,
%                framesep=2mm]{python}
% 	...		% 此处插入代码块,需要缩进
% \end{minted}
\begin{lstlisting}[language={java}]
public class Main {
    public static void main(String[] args)
    {
        System.out.println("Hello,World");
    }
}
\end{lstlisting}