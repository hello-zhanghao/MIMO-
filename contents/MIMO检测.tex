\part{MIMO检测}
\section{点对点MIMO检测}
\subsection{系统模型}
$$
y=Hx+z
$$

% \begin{enumerate}
%     \item  参考论文\cite{2019Massive} \par
%         MIMO综述,主要综述了关于MIMO检测的问题
%     \item 参考论文2
% \end{enumerate}
\subsection{经典检测方法}
% 假设接收端已知CSI $H^{UL}$,系统模型表示为:
% \begin{equation}
% y_{MAC}=H^{UL}x+z 简化为 y=Hx+z=H_1x_1+H_2x_2+\cdots+H_Kx_K+z
% \end{equation}
\subsubsection{MF检测}
\begin{equation}
\hat{x}_{MF}=H^Hy
\end{equation}
匹配滤波器,又称最大比合并,目标是最大化接收信噪比。
\begin{itemize}
    \item 优点:复杂度低,不需要矩阵求逆运算
    \item 缺点:对于病态矩阵(ill-conditioned),检测性能严重劣化
\end{itemize}
% \end{enum}
    
\subsubsection{ZF检测}
\begin{equation}
    \hat{x}_{ZF}=G_{ZF}y
\end{equation}
其中$G_{ZF}=(H^HH)^{-1}H^H$,目标是最大化接收信干噪比(received signal-to-interference ration)
\begin{itemize}
    \item 优点:性能优于MF(复杂度较MF高)
    \item 缺点:对忽略噪声的影响,与MF类似,会有噪声加强的副作用
\end{itemize}
\subsubsection{MMSE检测}
MMSE检测是优化问题的解:
\begin{equation}
\begin{aligned}
    G_{MMSE}&=\underset{G\in C^{N_t \times N_r}}{\arg min}\|x-Gy\|^2 \\
    &= \underset{G}{\arg min}E\{\|Gy-x\|^2\}=(H^HH+\sigma^2I_{Nt})^{-1}H^H
\end{aligned}
\end{equation}
于是,MMSE检测表达式为:
\begin{equation}
    \hat{x}_{MMSE}=G_{MMSE}y
\end{equation} \\

\emph{regularized channel inversion}
\begin{equation}
    \underline{H}=\begin{bmatrix}
        H \\
        \sigma^2I_{N_t}
        \end{bmatrix} and  \ \ 
        \underline{y}=
        \begin{bmatrix}
        y \\
        0_{N_t\times1}
        \end{bmatrix} 
\end{equation}
Then
\begin{equation}
    G_{MMSE}=(\underline{H}^H\underline{H})^{-1}\underline{H}^H
\end{equation}
\begin{itemize}
    \item 优点:有效解决了ZF和MF导致的噪声加强问题,能够在噪声功率较大(信噪比较低时)时获得较大的性能增益
    \item 缺点:需要对噪声功率谱密度的先验信息,且需要矩阵求逆
\end{itemize}

\subsubsection{ZF-SIC检测}


\subsubsection{MMSE-SIC检测}
\subsubsection{总结}
最早的大规模MIMO检测器是在2008年由Vardhan et al. \cite{Vishnu2008A}提出的,是基于最大似然搜索的。紧接着,研究者提出了使用局部搜索和置信传播提出了近似ML的检测器。但是随着天线数目的增加,矩阵求逆操作的复杂度呈指数增长,计算复杂度急剧上升,为了解决这个问题,2013年,
Wu et al.\cite{2013Approximate} 提出了基于近似求逆方法的上行检测器。

\subsection{Massive MIMO 检测器}
\subsubsection{基于近似求逆的线性检测器}
\subsubsection{Newton Iteration Method}
\subsubsection{Gauss-Seidel Method}
\subsubsection{Successive Over-Relaxation}
\subsubsection{Jacobi Method}
\subsubsection{Richardson Method}
\subsubsection{Conjugate Gradients Method}
\subsubsection{Lanczos Method}
\subsubsection{Residual Method}
\subsubsection{Coordinate Descent Method}
\subsection{基于局部搜索的检测器}
\subsubsection{Likelihood Ascent Search}
\subsubsection{Reactive Tabu Search}
\subsection{基于置信传播的检测器}
\subsection{BOX Detection}
\subsection{Sparsity Based Algorithms}
\subsection{小尺度MIMO检测器在大尺度MIMO检测中的应用}
\subsubsection{Successive Interference Cancellation}
\subsubsection{Lattice Reduction-Aided Algorithms}
\subsubsection{Sphere Decoder}
\subsection{基于AI的MIMO检测方法}
